\documentclass[journal]{IEEEtran}

% *** CITATION PACKAGES ***
\usepackage[style=ieee]{biblatex}
\bibliography{example.bib}

% *** MATH PACKAGES ***
\usepackage{amsmath}

% *** PDF, URL AND HYPERLINK PACKAGES ***
\usepackage{url}
% correct bad hyphenation here
\usepackage{graphicx}  %needed to include png, eps figures
\usepackage{float}  % used to fix location of images i.e.\begin{figure}[H]

\begin{document}
\title{Review of ``Sparsity and Cosparsity for audio declipping: a flexible non convex approach''}
\author{Pierre-Antoine Comby}% <-this % stops a space


% make the title area
\maketitle

% As a general rule, do not put math, special symbols or citations
% in the abstract or keywords.
\begin{abstract}
  ptetzknfln
\end{abstract}
\section{Introduction }
Include a title and one or two paragraphs describing what you plan to do. Tell what interviews, site visits, or other activity you plan. Be specific if you can. Include  one good reference you plan to used. This is an example of how to include a citation \cite{williams1989style}.  \\.

\section {Background}
Give a background on your topic. Include references.

\section{Results}
Results of your interviews or observations. Use information and/or quotes from your interview or observations.

\section{Discussion}
Your comments or evaluation of interview or observations

\section{Summary}
Summary of your paper.

\appendix
The Appendix should contain the name, position, and company (or other relevant information) for the person(s) you interviewed or the places you visited. For interviews, include your list of questions and indicate if the interview was in person, by phone, or by email. (In-person interviews are best, but may not be available for some topics.) Include the person's answers. (A summary is ok.) If you identify the person fully and quote extensively from the interview in the body of your paper you do not have to include the appendix. The Appendix does not count toward the 4000 word requirement.


\section{Summary}
Summarize your paper.




\printbibliography

% that's all folks
\end{document}
